# Week 5 Quiz
%-------------------------------------------------------------------------%
\subsection{Question 1}

Load the weaving data with the commands:

```{r}
data(warpbreaks)
```

Fit an ANOVA model where the outcome is the number of breaks. Fit an ANOVA model
including the wool and tension variables as covariates. What are the dgrees of
freedom for the tension variable and what is the F-statistic for tension after
accounting for the variation due to wool?

```{r}
warpANOVA <- aov(warpbreaks$breaks ~ warpbreaks$wool + warpbreaks$tension)
summary(warpANOVA)
```

**The degrees of freedom for tension is 2 and the F-statistic is 7.537.**
%-------------------------------------------------------------------------%
\subsection{Question 2}

Suppose that the probability an event is true is 0.2. What are the log odds of that event?

```{r}
log( (0.2 / (1 - 0.2)) )
```

**-1.3863**
%-------------------------------------------------------------------------%
\subsection{Question 3}

Load the [horseshoe crab](http://en.wikipedia.org/wiki/Horseshoe_crab) data using the commands:

```{r}
library(glm2)
data(crabs)
```

Fit a Poisson regression model with the number of Satellites as the outcome and
the width of the female as the covariate. What is the multiplicative change in
the expected number of crabs for each additional centimeter of width?

```{r}
plot(crabs$Width, crabs$Satellites,
     pch=19, col="darkgrey")
crabsGLM <- glm(crabs$Satellites ~ crabs$Width, family="poisson")
#abline(crabs$Width, crabsGLM$fitted.values, col="blue", lwd=3)
#abline(0, 0.16405, col="blue", lwd=3)
#lines(crabs$Width, crabsGLM$fitted.values, col="blue", lwd=3)

### His use of `abline` doesn't match w/ the API???
### How do we really go about using `abline`?

summary(crabsGLM)

# this is it: `exp` of the coefficient
exp(crabsGLM$coefficients[[2]])
```

~~0.1640~~
**1.1782**
_And/but: just a guess! (How do I arrive at the real answer?)_
%-------------------------------------------------------------------------%
\subsection{Question 4}

Load the horseshoe crab data using the commands:

```{r}
# loaded in question 3
```

What is the expected number of Satellites for a female of width 22cm?

```{r}
# really that easy?
22 * 0.1640

## Dave: "No. This is it..."
 exp(crabsGLM$coefficients[[1]]) * exp(crabsGLM$coefficients[[2]] * 22)
```

~~3.6080~~
**1.3556**
_And/but: just a guess! (How do I arrive at the real answer?)_

%-------------------------------------------------------------------------%
\subsection{Question 5}

Load the school absenteeism data set and fit a linear model relating the log of
the number of days absent to the other variables with the commands:

\begin{verbatim}
library(MASS)
data(quine)
# dot means all variables
lm1 <- lm(log(Days+2.5) ~ ., data=quine)
\end{verbatim}

Use the `step()` function in R to perform model selection using default
parameters. What variables remain in the model after model selection?

```{r}
aic <- step(lm1)
aic$model
```

**Eth, Age**
