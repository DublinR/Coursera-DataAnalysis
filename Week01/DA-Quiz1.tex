\documentclass[]{article}

\usepackage{framed}
\usepackage{amsmath}


%opening
\title{Data Analysis Week 1 Homework}
\author{}

\begin{document}

\maketitle



\section*{Quiz 1}
%--------------------------------------------------------------------------%

\subsection*{Question 1}
A web administrator is examining the web log file which provides information about people who visited his site. In the log file, the administrator sees 
\begin{itemize}
\item a value for the internet protocol (IP) address of ``127.0.0.1", \item a user ID of ``frank" for the individual accessing the file and \item ``2326", measured in bytes, for the size of the file returned to the user.
\end{itemize} Which of the following are true? 

\vspace{0.4cm}
\noindent \textbf{Options:}
\begin{enumerate}
\item The IP address is a quantitative variable and the user ID and bytes returned are qualitative variables.
\item The IP address and the user ID are qualitative variables and the number of bytes returned is quantitative.
\item The IP address, user ID, and bytes returned are all qualitative variables.
\item The IP address, user ID, and bytes returned are all quantitative variables.
\end{enumerate}

\vspace{0.2cm}
\noindent \textbf{Required:}  Knowledge of quantitative and qualitative variables

\begin{itemize}
\item \textbf{Quantitative}  : Expression some numeric value, a count, a measure or a ranking
\item \textbf{Qualitiative} : names, categories, descriptions, codes eg P.I.N and telephone numbers
\end{itemize}


{
\small
\noindent \textbf{Useful Sites}
\begin{itemize}
\item Here is a description of a web log file in common log format:
\begin{verbatim}
 http://en.wikipedia.org/wiki/Common_Log_Format. 
\end{verbatim}
\item Here is a description of an IP address:
\begin{verbatim}
http://en.wikipedia.org/wiki/IP_address
\end{verbatim}
\end{itemize}
}
% **The IP address and the user ID are qualitative variables and the number of bytes returned is quantitative.**
%--------------------------------------------------------------------------%
\newpage
\subsection*{Question 2}
Suppose that random variable X follows a Poisson distribution with rate parameter \textbf{L}. If we increase the value of \texttt{L}, which of the following is true?


\bigskip
\noindent\textbf{Options:}
\begin{enumerate}
\item The spread increases but the center remains unchanged.
\item Both the spread and the center increase.
\item The center increases but the spread decreases.
\item The spread increases but the center decreases.
\end{enumerate}
\vspace{0.5cm}
\noindent \textbf{Comments:}
\begin{itemize}
\item center - i.e. the measures of centrality, such as mean and median.
\item spread - i.e. measures of disperion, such as variance and range.
\end{itemize} 

\vspace{0.5cm}
\noindent \textbf{Exercise}
\begin{itemize}
\item Generate 100 random numbers from the Poisson distribution - specifying a value for \texttt{lambda} (i.e. what the rate parameter is called when using \texttt{R}) .
\item Compute the mean and variance for this set of numbers.
\item Repeat the process a few times, each time increasing the value of lambda.
\end{itemize}
\begin{framed}
\begin{verbatim}
#generate three data sets
X1 <- rpois(100, lambda= 4)
X2 <- rpois(100, lambda= 8)
X3 <- rpois(100, lambda= 18)

#Now get the mean and variance for each data set
mean(X1);var(X1)
mean(X2);var(X2)
mean(X3);var(X3)
\end{verbatim}
\end{framed}

% ~~The center increases but the spread remains unchanged.~~
%**Both the spread and the center increase.**

%--------------------------------------------------------------------------%
\newpage
\subsection*{Question 3}
Run the following commands to create a data frame in R with measurements for 30 men describing their height in centimeters, weight in kilograms, and a logical indicator for whether they have a daughter or not.

\begin{framed}
\begin{verbatim}
set.seed(31)
heightsCM <- rnorm(30,mean=188, sd=5)
weightsK <- rnorm(30,mean=84,sd=3)
hasDaughter <- sample(c(TRUE,FALSE),size=30,replace=T) 
dataFrame <- data.frame(heightsCM,weightsK,hasDaughter)
\end{verbatim}
\end{framed}
\begin{itemize}
\item Subset the data frame to only the individuals that are greater than 188 centimeters tall. 

\item Assign this subset to a data frame called \textit{\textbf{dataFrameSubset}}. 
Then run the following command to get the average weight among this subset of men in the data. 
\end{itemize}
\begin{framed}
\begin{verbatim}
mean(dataFrameSubset$weightsK)
\end{verbatim}
\end{framed} 
What is the value that is produced?

\begin{framed}
\begin{verbatim}
dataFrameSubset <- subset(dataFrame, heightsCM > 188)
mean(dataFrameSubset$weightsK)
\end{verbatim}
\end{framed}
%**82.40639**
%--------------------------------------------------------------------------%
\newpage
\subsection*{Question 4}


Run a command to generate 100 Cauchy random variables with default parameters and assign them to a vector \textit{\textbf{cauchyValues}} immediately after running the command

\begin{framed}
\begin{verbatim}
set.seed(41)
cauchyValues <- rcauchy(100)
\end{verbatim}
\end{framed}

\noindent Then run a command to sample 10 values with replacement from \textit{\textbf{cauchyValues}} immediately after running the command.

\noindent \textit{\textbf{Note}: It is critical that you run the \texttt{set.seed()} commands immediately before the commands to perform the data generation and sampling or you will get the wrong answer.}

\begin{framed}
\begin{verbatim}
#Run together - not separately
set.seed(415)
sample(cauchyValues, 10, replace=TRUE)
\end{verbatim}
\end{framed}

\noindent What are the first three values of the resulting sample?
 


%**0.8084719, -1.1122863, 0.3716671**
\subsubsection*{The Cauchy Distribution : Uses}
(Source: \textit{\textbf{http://www.vosesoftware.com/ModelRiskHelp}})\\
(\textit{Side Panel - Click on ``more", ``distribution" and then ``list of distributions.})
\\
\vspace{0.5cm}
The Cauchy distribution  is used in mechanical and electrical theory, physical anthropology and measurement and calibration problems. In physics it is usually called a Lorentzian distribution, where it is the distribution of the energy of an unstable state in quantum mechanics. It is also used to model the points of impact of a fixed straight line of particles emitted from a point source.

%--------------------------------------------------------------------------%
%Ready
\newpage
\subsection*{Question 5}
\begin{itemize}
\item We take a random sample of individuals in a population and identify whether they smoke and if they have cancer. 
\item We observe that there is a strong relationship between whether a person in the sample smoked or not and whether they have lung cancer.
\item  We claim that the smoking is related to lung cancer in the larger population.
\end{itemize}
%**This is an example of an inferential data analysis.**

\noindent \textbf{Options}
\begin{enumerate}
\item This is an example of a mechanistic data analysis.
\item This is an example of an inferential data analysis.
\item This is an example of an descriptive data analysis.
\item This is an example of a predictive data analysis.
\end{enumerate}

\noindent \textbf{Comments}
\begin{itemize}
\item Inferential data analysis related to making statements about a population based on statistics computed from representative samples.

\item There is such a thing as ``\textit{\textbf{exploratory data analysis}}" - which is essentially a collation of descriptive statistics and summary graphic to inform end users about a sample. \\ Importantly - suggesting hypotheses about a population is not part of this process.

\item A predictive model is specifically about future events, predicted from currently available data. Making a statement about something currently happening is not predictive - but rather retrospectivally looking at contributing factors.

\item Mechanistic?  Causal? - never heard of those.
\end{itemize}
%--------------------------------------------------------------------------%
\newpage
\subsection*{Question 6}
%READY
Suppose that we collect data on every goal scored in the Spanish Primera Division in the 2011/2012 and 2012/2013 seasons. (Source: \texttt{http://soccernet.espn.go.com})

\begin{itemize}
\item We use the data from 2011/2012 to build a model to predict the number of goals scored in 2012/2013.
\item What is the complete list of labels that apply to this data set?
\end{itemize} 

%**Census, prediction, longitudinal**
\textbf{Options}
\begin{enumerate}
\item Prediction and cross-sectional
\item Prediction, inferential, and longitudinal
\item Only prediction
\item Census, prediction, longitudinal
\end{enumerate}

\textbf{Comments}
\begin{itemize}
\item Census - ``......data on \textbf{every} goal scored.."
\item Prediction - ``........ \textbf{predict} the number.."
\item Prediction and Inferential - very different purposes. 
\item Defining characteristic of \textbf{cross sectional studies} are that they take place at a single point in time.
\item Defining characteristic of \textbf{longtitudinal studies} are that they take place over multiple periods of time, collecting data at every period.
\end{itemize}
%--------------------------------------------------------------------------%
\newpage
\subsection*{Question 7}
%READY
What are the three characteristics of tidy data?

\begin{itemize}
\item ``\textit{\textbf{Tidy data}}" by Hadley Wickham (RStudio)
\item Submission to Journal of Statistical Software
\item (http://vita.had.co.nz/papers/tidy-data.pdf)
\end{itemize}
Three Principles from Hadley Wickham's paper
\begin{itemize}
\item[1.] Each variable forms a column, 
\item[2.] Each observation forms a row, 
\item[3.] Each table/file stores data about one kind of observation.
\end{itemize}
%--------------------------------------------------------------------------%
\newpage
\subsection*{Question 8}
%READY
Which of the following are components of data processing that should be recorded for use in later data analyses?

% **All of the above**
\bigskip
\textbf{Options}

\begin{enumerate}
%\item Which of the following are components of data processing that should be recorded for use in later data analyses?
\item Which data files were merged together to create the processed data (\textbf{\textit{sensible}})
\item What day the data were downloaded from a website (\textbf{\textit{sensible}})
\item The software used to process the raw data. (\textbf{\textit{sensible}})
\item All of the above (!!)
\end{enumerate}
%--------------------------------------------------------------------------%
%Ready
\newpage
\subsection*{Question 9}
When writing about data, what does it mean when we write: \[ X | Y = y\]

\begin{itemize}
\item $X$ and $Y$ are random variables. Random variables are conventionally denoted with capital letters. Small letters conventionally denote the location in the equation for specific values.
(i.e. X : Height , x = 1.82 meters )
\item Suppose $X$ was height and $Y$ was weight. Also let us suppose that $y$ is 14 stone.
\item We might understand the expression as \textbf{\emph{Heights of people who weigh 14 stone}}?
\item The likely distribution of heights would be different from people who are 10 stone for example.
\item That is to say - the distribution of X depends on the value of Y. The converse case  $Y | X = x$ can also be true.
\end{itemize}

%**We are referring to the random variable X when we know the random variable Y has value y. 
%The distribution of this variable may be different than the distribution of the variable X when Y is also random.**
%--------------------------------------------------------------------------%
\newpage
\subsection*{Question 10}
%READY
\begin{itemize}
\item Suppose we take a sample of people in Baltimore and observe that younger people have taken more Coursera courses. 
\item We use an inferential data analysis to show a relationship between age and Coursera courses. 
\item Which of the following statements are true based only on our analysis?
\end{itemize}
\noindent \textbf{Options}
\begin{enumerate}
\item The reason that younger people take more Coursera courses is that they understand technology better. \\(\textit{\textbf{Speculative Generalization - can't do this} : Nothing in study about general use of technology.})			
\item If we took a census of all people in Baltimore, we would expect to see that the younger a person was, the more likely they were to have taken a Coursera course.		
\item When comparing two people in Baltimore, the one who is younger will always have taken more Coursera courses. (\textbf{\textit{Ecological Fallacy}})			
\item If we took a census of all people in \textbf{Boston}, we would expect to see that the younger a person was, the more likely
 they were to have taken a Coursera course.(\textbf{\textit{Wrong Town - study not performed in Boston}})
\end{enumerate}
\vspace{0.5cm}
\noindent \textbf{The Ecological Fallacy}\\
The Ecological Fallacy is a situation that can occur when a researcher or analyst makes an inference about particular individuals based on aggregate data for a group. 
(Source: \textit{http://jratcliffe.net/research/ecolfallacy.htm})

\bigskip
(Also check out ``\textit{\textbf{Simpson's paradox}}".)
%**If we took a census of all people in Baltimore, we would expect to see that the younger a person was, the more likely they were to have taken a Coursera course.**
\end{document}
