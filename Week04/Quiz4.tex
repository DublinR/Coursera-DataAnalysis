% Week 4 Quiz

\documentclass[12pt]{article}

%opening
\usepackage{framed}
\usepackage{amsmath}
\usepackage{amssymb}
\usepackage{graphicx}
\begin{document}


\section{Data Analysis : Week 4 Quiz}
%--------------------------------------

%--------------------------------------------------------------------%
\subsection*{Question 1}
Load the movies data available from the course website here: 

https://spark-public.s3.amazonaws.com/dataanalysis/movies.txt

Fit a linear regression model by least squares where the Rotten Tomatoes score
is the outcome and the box office gross is the only covariate. What is the
regression coefficient for the slope and it's interpretation?

\begin{verbatim}
setwd('~/Desktop/coursera-data-analysis/quizzes')
download.file('https://spark-public.s3.amazonaws.com/dataanalysis/movies.txt', 'movies.txt', method='curl')
movies <- read.delim('movies.txt')

names(movies); head(movies)

plot(movies$box.office, movies$score)
moviesLM <- lm(movies$score ~ movies$box.office)
lines(movies$box.office, moviesLM$fitted, col="red", lwd=3)

moviesLM$coefficients
\end{verbatim}

%**The regression coefficient is 0.09676. The interpretation is that an increase
%of one million dollars in box office gross is associated with a 0.09676 increase
%in Rotten Tomatoes Score.**
%--------------------------------------------------------------------%
\subsection*{Question 2}

Load the movies data available from the course website here: 

\begin{verbatim}
https://spark-public.s3.amazonaws.com/dataanalysis/movies.txt
\end{verbatim}

Fit a linear regression model by least squares where the Rotten Tomatoes score
is the outcome and the box office gross is the only covariate. What is the 90\%
confidence interval for the intercept term and what can you deduce from the 90\%
confidence interval?

\begin{framed}
\begin{verbatim}
confint(moviesLM,level=0.9)
\end{verbatim}
\end{framed}
**The 90% confidence interval for the intercept is (47.52, 52.63). If we
repeated this study 100 times, we would expect our calculated interval to cover
the true value on average 90% of the time.**
%--------------------------------------------------------------------%
\newpage
\subsection*{Question 3}

Load the movies data available from the course website here: 

\begin{verbatim}
https://spark-public.s3.amazonaws.com/dataanalysis/movies.txt 
\end{verbatim}


Fit a linear regression model by least squares where the Rotten Tomatoes score
is the outcome and box office gross and running time are the covariates. What is
the value for the regression coefficient for running time? How is it interpreted?

\begin{framed}
\begin{verbatim}


plot(movies$running.time, movies$score)
moviesLM2 <- lm(movies$score ~ movies$box.office + movies$running.time)

moviesLM2$coefficients
\end{verbatim}
\end{framed}
% lines(movies$running.time, moviesLM2$fitted, col="red", lwd=3)


%**The coefficient is 0.12752. That means that if two movies have the same box
%office gross, an increase of one minute in running time is associated with an
%average increase of 0.12752 in score.**
%--------------------------------------------------------------------%
\newpage
\subsection*{Question 4}

Load the movies data available from the course website here: 

https://spark-public.s3.amazonaws.com/dataanalysis/movies.txt 

Fit a linear regression model by least squares where the Rotten Tomatoes score
is the outcome and box office gross and running time are the covariates. Is
running time a confounder for the relationship between Rotten Tomatoes score and
box office gross? Why or why not?

```{r}
# already have the data...
# already have `moviesLM2`

q4LM <- lm(movies$score ~ movies$box.office + movies$running.time)

par(mfrow=c(3,1))
plot(movies$box.office, movies$score)
lines(movies$box.office, lm(movies$score ~ movies$box.office)$fitted, col="red", lwd=3)
plot(movies$running.time, movies$score)
lines(movies$running.time, lm(movies$score ~ movies$running.time)$fitted, col="red", lwd=3)
plot(movies$running.time, movies$box.office)
lines(movies$running.time, lm(movies$box.office ~ movies$running.time)$fitted, col="red", lwd=3)

q4LM$coefficients
summary(q4LM)
```
**Yes running time is a confounder. It is correlated both with the Rotten
Tomatoes score and the box office gross.**
%--------------------------------------------------------------------%
\newpage
\subsection*{Question 5}

Load the movies data available from the course website here: 

https://spark-public.s3.amazonaws.com/dataanalysis/movies.txt 

Fit a linear regression model by least squares where the Rotten Tomatoes score
is the outcome and box office gross and running time are the covariates. Make a
plot of the movie running times versus movie score. Do you see any outliers? If
you do, remove those data points and refit the same regression (Rotten Tomatoes
score is the outcome and box office gross and running time are the covariates).
What do you observe?

\begin{verbatim}
# already have the data...
# already have `moviesLM2`

par(mfrow=c(1,1))
plot(movies$running.time, movies$score)

movies.shorter <- movies[movies$running.time < 200,]

moviesLM2b <- lm(movies.shorter$score ~ movies.shorter$box.office + movies.shorter$running.time)

plot(movies.shorter$running.time, movies.shorter$score)
lines(movies.shorter$running.time, moviesLM2b$fitted, col="red", lwd=3)
# again: a scribble: WTF?

summary(lmNoAdjust)
# look for P-values
\end{verbatim}

**Yes there are two outliers. After removing them and refitting the regression
line, the running time coefficient has a larger magnitude and is more
statistically significant.**
%--------------------------------------------------------------------%
\newpage
\subsection{Question 6}

Load the movies data available from the course website here: 

https://spark-public.s3.amazonaws.com/dataanalysis/movies.txt 

Fit a linear regression model by least squares where the Rotten Tomatoes score
is the outcome and running time and box office gross are covariates. What is the
P-value for running time and how is it interpreted?

```{r}
# already have the data...
# have `moviesLM2` from above
q6LM <- lm(movies$score ~ movies$running.time + movies$box.office)
summary(q6LM)
```
**The P-value is 0.0187. It is the probability of observing a t-statistic as big
as, or larger than, the one we saw, if there was no relationship between Rotten
Tomatoes score and running time for a fixed box office gross.**

%--------------------------------------------------------------------%
\subsection{Question 7}

Load the movies data available from the course website here: 

https://spark-public.s3.amazonaws.com/dataanalysis/movies.txt 

Fit a linear model by least squares where Rotten Tomatoes score is the outcome
and the covariates are movie rating, running time, and an interaction between
running time and rating are the covariates. 

What is the coefficient for the
interaction between running time and the indicator/dummy variable for PG rating?

\begin{verbatim}
# already have the data...
moviesLMx <- lm(movies$score ~ movies$rating + 
  movies$running.time + movies$running.time*movies$rating)
\end{verbatim}
% **The coefficient is -0.6901.**
%--------------------------------------------------------------------%
\subsection{Question 8}

Load the movies data available from the course website here: 

https://spark-public.s3.amazonaws.com/dataanalysis/movies.txt 

Fit a linear model by least squares where Rotten Tomatoes score is the outcome
and the covariates are movie rating, running time, and an interaction between
running time and rating are the covariates. 

What is the estimated average change
in score for a PG movie for a one minute increase in running time?

\begin{verbatim}
# already have the data...
q8LM <- lm(movies$score ~ movies$rating + movies$running.time + movies$running.time*movies$rating)
## (running.time coefficient + PG:running.time)
\end{verbatim}
**0.4951**
%--------------------------------------------------------------------%

\subsection8{Question 9}

Load the data on number of breaks during weaving into R with the command:

\begin{verbatim}
data(warpbreaks)
\end{verbatim}

Fit a linear model where the outcome is the number of breaks and the covariate
is tension. What is a 95% confidence interval for the average difference in
number of breaks between medium and high tension?

\begin{verbatim}
warpLM <- lm (warpbreaks$breaks ~ warpbreaks$tension)
confint(warpLM, level=0.95)
summary(warpLM)
\end{verbatim}

%**(-3.23, 12.67)**
%(though I'll admit I don't know how I got here...)

%**LATER:** BLN says: go back to the "Regression with Factor Variables" lecture
%and zero in on "Question 2" ("What is the average difference in rating b/w PG-13
%and R-rated movies?") -- it's in there as `relevel`.
%--------------------------------------------------------------------%
\subsection*{Question 10}

Based on this graph is there a significant association between organic food
sales and autism rates? Would you believe this association could be used to
reduce autism rates? Why or why not?

![Figure for Question 10](http://i.imgur.com/1WZ6h.png)

**There is a statistically significant association. We may be skeptical this
association could be used to reduce autism rates, since there are many possible
explanations for the association that do not involve a direct relationship
between organic foods and autism.**
